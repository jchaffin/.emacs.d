\documentclass{article}
\usepackage[usenames]{color}
\pagestyle{empty}
% do not remove
% The settings below are copied from fullpage.sty
\setlength{\textwidth}{\paperwidth}
\addtolength{\textwidth}{-3cm}
\setlength{\oddsidemargin}{1.5cm}
\addtolength{\oddsidemargin}{-2.54cm}
\setlength{\evensidemargin}{\oddsidemargin}
\setlength{\textheight}{\paperheight}
\addtolength{\textheight}{-\headheight}
\addtolength{\textheight}{-\headsep}
\addtolength{\textheight}{-\footskip}
\addtolength{\textheight}{-3cm}
\setlength{\topmargin}{1.5cm}
\addtolength{\topmargin}{-2.54cm}
\usepackage{fontspec}
\usepackage{unicode-math} % Support unicode input
\setmainfont{STIX Two Text} % Default Font
\setmathfont{STIX Two Math}[StylisticSet=02,StylisticSet=08] % Math font
\setmonofont[Scale=MatchUppercase]{DejaVu Sans Mono} % Monospace Font
\usepackage{cabin} % Sans Serif font
\usepackage{soul}
\usepackage{marvosym}
\newcommand{\vertex}{\node[circle,draw,radius=1cm]}
\usepackage{tikz}
\usepackage{forest}
\forestset{% node styles
  nice nodes/.style={
    for tree={ inner sep=1 pt, s sep=12pt, fit=band }
  },
  fairly nice empty nodes/.style={
    delay={
      where content={}{
        shape=coordinate,
        for parent={
          for children={anchor=north}
        }
      }
    }
  },
  pretty nice empty nodes/.style={
    for tree={
      calign=fixed edge angles,
      parent anchor=children,
      delay={ if content={}{ inner sep=0pt, edge path={
            \noexpand\path [\forestoption{edge}] (!u.parent anchor) --
            (.children)\forestoption{edge label} ;
          }
        }{}
      }
    }
  },
  perfectly nice empty nodes/.style={
    for tree={ calign=fixed edge angles, parent anchor=south },
    before typesetting nodes={
      where content={}{ text width=.001pt, inner sep=0pt,
        before drawing tree={ shape=coordinate, typeset node },
        for parent={ for children={ anchor=north } }
      }{}
    }
  },
  circle nodes/.style={
    for tree={circle,draw,l sep=2mm}%
  },
  default preamble = {
    for tree={
      text height=1em, inner ysep=0.5ex, s sep=1em, edge={semithick},
      text depth={}, l sep=1em
    }
  }
}

\usepackage[algoruled,linesnumbered]{algorithm2e}
\newcommand{\nosemic}{\SetEndCharOfAlgoLine{\relax}} % Drop semi-colon ;
\newcommand{\dosemic}{\SetEndCharOfAlgoLine{\string;}} % Reinstate
\newcommand{\pushline}{\Indp}% Indent
\newcommand{\popline}{\Indm\dosemic}% Remove indent
% Default Keywords
\SetKwInOut{Input}{Input}
\SetKwInOut{Output}{Result}
\SetKwProg{proc}{Procedure}{}{}
\SetKwComment{Comment}{ $\triangleright$\ }{}
\SetKwProg{Fn}{Function}{:}{end}
% 'for' loop Macros
\newcommand{\forcond}[3]{$#1=#2$ \KwTo $#3$}
\newcommand{\forcondi}[2]{\forcond{i}{#1}{#2}}
\newcommand{\forcondj}[2]{\forcond{j}{#1}{#2}}
\SetFuncSty{\scshape}
% from Russell's Artificial intelligence A Modern Approach
\SetKwFunction{InitialState}{Initial-State}
\SetKwFunction{State}{State}
\SetKwFunction{PathCost}{Path-Cost}
\SetKwFunction{ChildNode}{Child-Node}
\SetKwFunction{Actions}{Actions}
\SetKwFunction{GoalTest}{Goal-Test}
\SetKwFunction{Solution}{Solution}
% Data
\SetKwData{node}{node}
\SetKwData{frontier}{frontier}
\SetKwData{explored}{explored}
% Data Structures
\SetKwFunction{Empty}{Empty?}
\SetKwFunction{Pop}{Pop}
\SetKwFunction{Push}{Push}
\SetKwBlock{Loop}{loop do}{}
