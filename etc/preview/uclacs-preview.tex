\documentclass{article}
\usepackage[usenames]{color}
\pagestyle{empty}
% do not remove
% The settings below are copied from fullpage.sty
\setlength{\textwidth}{\paperwidth}
\addtolength{\textwidth}{-3cm}
\setlength{\oddsidemargin}{1.5cm}
\addtolength{\oddsidemargin}{-2.54cm}
\setlength{\evensidemargin}{\oddsidemargin}
\setlength{\textheight}{\paperheight}
\addtolength{\textheight}{-\headheight}
\addtolength{\textheight}{-\headsep}
\addtolength{\textheight}{-\footskip}
\addtolength{\textheight}{-3cm}
\setlength{\topmargin}{1.5cm}
\addtolength{\topmargin}{-2.54cm}
\usepackage{fontspec}
\usepackage{unicode-math} % Support unicode input
\setmainfont{STIX Two Text} % Default Font
\setmathfont{STIX Two Math}[StylisticSet=02,StylisticSet=08] % Math font
\setmonofont[Scale=MatchUppercase]{DejaVu Sans Mono} % Monospace Font
\usepackage{cabin} % Sans Serif font
\usepackage{soul}
\usepackage{marvosym}
\usepackage{booktabs}
\usepackage{rail}
\newcommand{\vertex}{\node[circle,draw,radius=1cm]}
\usepackage{tikz}
\usetikzlibrary{automata, positioning, arrows}
  \tikzset{
    ->, % makes the edges directed
    node distance=3cm, % specifies the minimum distance between two nodes. Change if necessary.
    every state/.style={thick}, % sets the properties for each ’state’ node
    initial text=$ $, % sets the text that appears on the start arrow
  }
\usepackage{circuitikz}
\railoptions{-t}
\usepackage{forest}
\forestset{
  circle nodes/.style{%
    for tree={circle,draw, l sep=2mm}%
  },
  default preamble={
    for tree={
      text height=1em, inner ysep=0.5ex, s sep=1em, edge={semithick},
      text depth={}, l sep=1em
    }
  }
}
\usepackage[algoruled,linesnumbered]{algorithm2e}
% Default Keywords
\SetKwInOut{Input}{Input}
\SetKwInOut{Output}{output}
\SetKwProg{proc}{Procedure}{}{}
\SetKwComment{Comment}{ $\triangleright$\ }{}
\SetKwProg{Fn}{Function}{:}{end}
\newcommand{\nosemic}{\SetEndCharOfAlgoLine{\relax}} % Drop semi-colon ;
\newcommand{\dosemic}{\SetEndCharOfAlgoLine{\string;}} % Reinstate
\newcommand{\pushline}{\Indp}% Indent
\newcommand{\popline}{\Indm\dosemic}% Remove indent
% 'for' loop Macros
\newcommand{\forcond}[3]{$#1=#2$ \KwTo $#3$}
\newcommand{\forcondi}[2]{\forcond{i}{#1}{#2}}
\newcommand{\forcondj}[2]{\forcond{j}{#1}{#2}}
\SetFuncSty{textsc}
% from Russell's Artificial intelligence A Modern Approach
\SetKwFunction{InitialState}{Initial-State}
\SetKwFunction{State}{State}
\SetKwFunction{PathCost}{Path-Cost}
\SetKwFunction{ChildNode}{Child-Node}
\SetKwFunction{Actions}{Actions}
\SetKwFunction{GoalTest}{Goal-Test}
\SetKwFunction{Solution}{Solution}
\SetKwFunction{Insert}{Insert}
% Data
\SetKwData{node}{node}
\SetKwData{frontier}{frontier}
\SetKwData{explored}{explored}
\SetKwData{problem}{problem}
\SetKwData{action}{action}
% Data Structures
\SetKwFunction{Empty}{Empty?}
\SetKwFunction{Pop}{Pop}
\SetKwFunction{Push}{Push}
\SetKwBlock{Loop}{loop do}{}
