\documentclass{article}
\usepackage[usenames]{color}
\pagestyle{empty}
% do not remove
% The settings below are copied from fullpage.sty
\setlength{\textwidth}{\paperwidth}
\addtolength{\textwidth}{-3cm}
\setlength{\oddsidemargin}{1.5cm}
\addtolength{\oddsidemargin}{-2.54cm}
\setlength{\evensidemargin}{\oddsidemargin}
\setlength{\textheight}{\paperheight}
\addtolength{\textheight}{-\headheight}
\addtolength{\textheight}{-\headsep}
\addtolength{\textheight}{-\footskip}
\addtolength{\textheight}{-3cm}
\setlength{\topmargin}{1.5cm}
\addtolength{\topmargin}{-2.54cm}


\usepackage{amsmath}
\usepackage{stmaryrd}
\usepackage{fontspec}
\usepackage{bussproofs}
\def\extraVskip{8pt}

\usepackage{unicode-math}
\usepackage{cabin}
\usepackage{xpatch}

\setmainfont{STIX Two Text}
\setmathfont{STIX Two Math}[StylisticSet=02, StylisticSet=08]
\setmonofont[Scale=MatchUppercase]{DejaVu Sans Mono}

\usepackage{gb4e}
\noautomath
\usepackage{adjustbox}
\usepackage{tikz}

\usetikzlibrary{decorations,decorations.pathreplacing,calc,fit,positioning,shapes,arrows.meta}
\tikzset{
  exarrows/.style={%
    semithick,
    arrows={%
      -Stealth[
        scale=1,
        scale length=1,
        scale width=1
      ]
    }
  }
}
\tikzset{
    dc/.style = {rounded corners=1pt, semithick},
    case/.style = {o, semithick, rounded corners=1pt},
    mvt/.style = {rounded corners=1pt, semithick},
    agree/.style = {rounded corners=1pt, semithick, o},
    icon/.style = {pos=0.25, yshift=-0.7ex},
    arrow-label/.style = {font=\scshape\itshape},
    o/.style = {
        shorten >=#1,
        shorten <=#1,
        decoration={
            markings,
            mark= at position 0 with
                { \fill circle [xshift=#1,radius=#1]; } ,
            mark= at position 1 with
                { \fill circle [xshift=-#1,radius=#1]; } ,
        },
        postaction=decorate
    },
    o/.default=1.75pt
}
% In-line Tikz nodes.
\newcommand{\tkn}[2]{%
  ~\tikz[na]{\node[yshift=-0.2ex](#1){\hspace*{-0.5ex}#2};~}}
\tikzstyle{na} = [remember picture,
                  baseline=-0.2ex,
                  inner xsep=0pt,
                  inner ysep=4pt,
                  outer xsep=0ex]

% Bracket diagrams.
\newenvironment{tikzbracket}{\begin{tikzpicture}[overlay, remember
    picture]}{\end{tikzpicture}}
% Movement arrows.
\tikzset{
    pics/mvt/.style n args={2}{code={
        \draw [->, mvt] (#1.south) -- ++(0, -\ArrowHeight) -| (#2.south);
    }},
    pics/mvt-above/.style n args={2}{code={
        \draw [->, mvt] (#1.north) -- ++(0, \ArrowHeight) -| (#2.north);
    }},
    pics/mvt-label/.style n args={3}{code={
        \draw [->, mvt] (#1.south) -- ++(0, -\ArrowHeight) -| (#2.south)
            node [arrow-label] {#3};
    }},
    pics/mvt-above-label/.style n args={3}{code={
        \draw [->, mvt] (#1.north) -- ++(0, \ArrowHeight) -| (#2.north)
            node [arrow-label-below] {#3};
    }},
}

% Case lines.
\tikzset{
    pics/case/.style n args={2}{code={
        \draw [case] (#1.south) -- ++(0, -\ArrowHeight) -| (#2.south);
    }},
    pics/case-above/.style n args={2}{code={
        \draw [case] (#1.north) -- ++(0, \ArrowHeight) -| (#2.north);
    }},
    pics/case-label/.style n args={3}{code={
        \draw [case] (#1.south) -- ++(0, -\ArrowHeight) -| (#2.south)
            node [arrow-label] {#3};
    }},
    pics/case-nej/.style n args={2}{code={
        \draw [case] (#1.south) -- ++(0, -\ArrowHeight) -| (#2.south)
            node[icon]{\nej};
    }}
}
\tikzstyle{label} = [draw, thin, rounded corners=1pt, fill=white, inner
sep=4pt, yshift =-3pt, font=\small\itshape, text height=.7em]
\tikzstyle{arrow-label} = [font=\small\itshape, pos=0.25, yshift=0.9ex]
\newlength{\ArrowHeight}
\setlength{\ArrowHeight}{3ex}
\newlength{\BelowArrowSkip}
\setlength{\BelowArrowSkip}{3ex}

\DeclareDocumentCommand \tikzexsetup {} {%
  \tikzstyle{every picture}+=[
    remember picture, inner sep=0pt,
    baseline, anchor=base
  ]
}
% arg 1: optional strut; arg 2: node name; arg 3: node content
\DeclareDocumentCommand \ND {s m m} {%
  \tikzifinpicture{}{\tikz}\node(#2){\IfBooleanTF{#1}{\strut}{}#3};}
\usepackage[linguistics,edges]{forest}
\forestset{
  % Begin nice nodes
  nice nodes/.style={
    for tree={
      inner sep=1 pt,
      s sep=12pt,
      fit=band
    }
  },
  % Begin fairly nice empty nodes
  fairly nice empty nodes/.style={
    delay={
      where content={}{
        shape=coordinate,
        for parent={
          for children={anchor=north}
        }
      }
    }
  },
  % Begin pretty nice empty nodes
  pretty nice empty nodes/.style={
    for tree={
      calign=fixed edge angles,
      parent anchor=children,
      delay={
        if content={}{
          inner sep=0pt,
          edge path={
            \noexpand\path [\forestoption{edge}] (!u.parent anchor) --
            (.children)\forestoption{edge label} ;
          }
        }{}
      }
    }
  },
  % Begin perfectly nice empty nodes
  perfectly nice empty nodes/.style={
    for tree={
      calign=fixed edge angles,
      parent anchor=south
    },
    before typesetting nodes={
      where content={}{
        text width=.001pt,
        inner sep=0pt,
        before drawing tree={
          shape=coordinate,
          typeset node
        },
        for parent={
          for children={
            anchor=north
          }
        }
      }{}
    }
  },
  default preamble = {
    for tree={
      text height=1em, inner ysep=0.5ex, s sep=1em, edge={semithick},
      text depth={}, l sep=1em
    }
  }
}


%%%%%%%%%%%%%%%%%%%%%%%%%%%%%%%%%%%%%%%%%%%%%%%%%%
%% Commands
%%%%%%%%%%%%%%%%%%%%%%%%%%%%%%%%%%%%%%%%%%%%%%%%%%
% https://tex.stackexchange.com/questions/132783/how-to-write-checkmark-in-latex
\ifpdf
  \newcommand*{\boldcheckmark}{%
    \textpdfrender{
      TextRenderingMode=FillStroke,
      LineWidth=.5pt, % half of the line width is outside the normal glyph
    }{\checkmark}%
  }
\else
 \newcommand*{\boldcheckmark}{\textbf{$🗸$}}
\fi


\newcommand{\nl}{$\varnothing$}
\newcommand{\ml}[1]{\textsc{#1}}
% Features
\newcommand{\self}[1]{\feature{$\bullet$#1$\bullet$}}
\newcommand{\plusf}[1]{\feature{$+$#1$+$}}
\newcommand{\selfc}[1]{\self{#1} $🗸$}
\newcommand{\plusfc}[1]{\sout{\plusf{#1}}}
\newcommand{\selfs}[1]{\sout{\self{#1}}}
\newcommand{\plusfs}[1]{$[$\sout{\plusfeat{#1}}$[$}

\newcommand{\pagree}[1]{$[\star\pi$: #1 $*]$}
\newcommand{\ppagree}[1]{$[\pi$: #1$]$}
\newcommand{\nagree}[1]{$[\star\#$: #1 $*]$}
\newcommand{\nnagree}[1]{$[\#$: #1$]$}
\newcommand{\tns}[1]{Tns$_{\text{#1}}$}
\newcommand{\vP}[0]{\textit{v}P\xspace}
\newcommand{\littlev}[0]{\textit{v}\xspace}

% Data macros.
\newcommand{\objlang}[2][]{\textit{#2}\ifx&#1&\else\ `#1'\fi}
\newcommand{\ja}[0]{\ding{51}}
\newcommand{\gap}[0]{\rule[-1.25pt]{1.5em}{0.75pt}\,}
\newcommand{\ind}[1]{\ensuremath{_{#1}}}

% Features.
\newcommand{\feature}[1]{{\scshape [#1]}}
\newcommand{\mf}[1]{\feature{$\bullet$#1$\bullet$}}
\newcommand{\hf}[1]{\feature{$+$#1$+$}}
\newcommand{\agrf}[1]{\feature{$\star$#1$\star$}}

% Satisfied features.
\newcommand{\shf}[1]{\hf{#1}{\scriptsize\,\ja}}
\newcommand{\smf}[1]{\mf{#1}{\scriptsize\,\ja}}
\newcommand{\sagrf}[1]{\agrf{#1}{\scriptsize\,\ja}}

% Phrase structure.
\newcommand{\head}[1]{#1$^0$}
\newcommand{\comp}[1]{\mbox{[Comp, #1]}}
\newcommand{\xbar}[1]{$\overline{\textrm{#1}}$}

% Semantics.
\newcommand{\denotation}[1]{\denotationbase{\text{#1}}}
\newcommand{\denotationbase}[1]{\ensuremath{\left\lsem#1\right\rsem}}
\newcommand{\startfn}[0]{\ensuremath{\ .\ }}
\newcommand{\Agree}[0]{\textsc{Agree}\xspace}
\newcommand{\abar}[0]{\xbar{A}}
\def\vb{\text{ | }}
